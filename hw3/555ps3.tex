
%!TEX TS-program = xelatex
%!TEX encoding = UTF-8 Unicode

\documentclass[12pt]{article}
\usepackage{geometry}                % See geometry.pdf to learn the layout options. There are lots.
\geometry{letterpaper}                   % ... or a4paper or a5paper or ... 
\usepackage{amsmath}
\usepackage{graphicx}
\usepackage{amssymb}
\usepackage{tikz}
\usepackage{listings}
\lstset{language=Octave,
  basicstyle=\footnotesize,tabsize=3,
commentstyle=\color{blue},
morecomment=[l]{\#}}
\usetikzlibrary[topaths]
\usepackage{fontspec,xltxtra,xunicode}
\defaultfontfeatures{Mapping=tex-text}
\setromanfont[Mapping=tex-text]{Hoefler Text}
\setsansfont[Scale=MatchLowercase,Mapping=tex-text]{Gill Sans}
\setmonofont[Scale=MatchLowercase]{Andale Mono}
\newcount\mycount

\title{Comp 555 Homework 3}
\author{Elliott Hauser}
\begin{document}
\maketitle
\section{Problem 8.6}

The shortest common superstring of the 8 3-mers given in the problem is given in Figure~\ref{SCS}.  This is the shortest possible subsequence because, for a set like this one where the minimum Hamming distance between strings is 1, the minimum theoretical length of a superstring is the number of substrings plus 2 (i.e. the Hamming distance of the two end pieces).  For the distance to be less than this, one or more of the substrings would have to be identical to one another.  To be a longer superstring, more fragments would have to be incompatible with other substrings' prefixes or suffixes.

\begin{figure}[hb]
\begin{center}
\begin{tabular}{cccccccccc}
A&G&T&A&A&A&C&T&T&T\\
A&G&T\\
 & G&T&A\\
&&T&A&A\\
&&&A&A&A\\
&&&&A&A&C\\
&&&&&A&C&T\\
&&&&&&C&T&T\\
&&&&&&&T&T&T\\
\end{tabular}
\end{center}
\caption{The shortest common superstring of 8 3-mers.}
\label{SCS}
\end{figure}




The Hamiltonian path approach to this problem is shown in Figure~\ref{hamilpath}.  The path does not visit every edge in the graph (since the Hamiltonian path is defined as visiting every node once and only once, regardless of edges), and the superstring represented by the path is the same, AGTAAACTTT.



\begin{figure}[hb]
\centering
\begin{tikzpicture}[shorten >=1pt,->,scale=.75]
  \tikzstyle{vertex}=[circle,fill=black!25,minimum size=11pt,inner sep=0pt]

  \foreach \name/\x in {AGT/0,GTA/1.5,TAA/3,AAA/4.5,AAC/6,ACT/7.5,CTT/9,TTT/10.5}
    \node[vertex] (G-\name) at (\x,0) {$\name$};

  \foreach \from/\to in {AGT/GTA,GTA/TAA,TAA/AAA,AAA/AAC,AAC/ACT,ACT/CTT,CTT/TTT}
    \draw[thick] (G-\from) -- (G-\to);

%leftwards 1
\foreach \from/\to in{AAA/AGT,CTT/TAA,TTT/TAA,ACT/TAA}
  \draw[dashed] (G-\from) .. controls +(-90:2.5cm) and +(270:2.5cm) .. (G-\to);

%leftwards 2
%\foreach \from/\to in{}
 % \draw (G-\from) .. controls +(-100:2.5cm) and +(280:3cm) .. (G-\to);

%rightwards 1
\foreach \from/\to in{AGT/TTT}
  \draw[dashed] (G-\from) .. controls +(80:3cm) and +(-260:3cm) .. (G-\to);
  
 %rightwards 2
\foreach \from/\to in{TAA/ACT,GTA/AAC}
  \draw[dashed] (G-\from) .. controls +(80:2cm) and +(-260:2cm) .. (G-\to);

 %rightwards 3
\foreach \from/\to in{ACT/TTT,AAC/CTT,GTA/AAA,AGT/TAA,AAA/ACT}
  \draw[dashed] (G-\from) .. controls +(80:1.5cm) and +(-260:1.5cm) .. (G-\to);

\end{tikzpicture}
\caption{The Hamiltonian Path approach to finding the shortest common superstring.  The 8 node Hamiltonian path is shown in solid, while other edges are shown dashed.}
\label{hamilpath}
\end{figure}

\begin{figure}[hb]
\centering
\begin{tikzpicture}[shorten >=1pt,->, scale=.75]
  \tikzstyle{vertex}=[circle,fill=black!25,minimum size=11pt,inner sep=0pt]

  \foreach \name/\x in {AG/0,GT/1.5,TA/3,AA/4.5,AC/6,CT/7.5,TT/9}
    \node[vertex] (G-\name) at (\x,0) {$\name$};

  \foreach \from/\to in {AG/GT,GT/TA,TA/AA,AA/AC,AC/CT,CT/TT}
    \draw[thick] (G-\from) -- (G-\to);

  \foreach \from/\to in {AA/AA}
    \draw[thick]  (G-\from) .. controls +(115:1.5cm) and +(75:1.5cm) .. (G-\to);

\end{tikzpicture}
\caption{The Eulerian Path approach to finding the shortest common superstring.}
\label{eulerpath}
\end{figure}


The Eulerian Path approach to this problem is shown in Figure~\ref{eulerpath}.  There is only one possible Eulerian path where all nodes are balanced (have indegree=outdegree) except for up to two semibalanced nodes, in this case AG and TT. The Eulerian path corresponds to the same superstring we saw above, AGTAAACTTT.  
\clearpage
\section{Problem 8.7}
If we reframe this problem as a Eulerian path problem through a graph with digits as the nodes, and edges in the graph representing the 2-digit numbers, we can see in Figure~\ref{numgraph} that it forms a complete graph with 10 nodes.  This string could be generated by finding a Eulerian cycle (being a complete graph, all nodes are balanced) in the graph.This means that there are many strings of the minimal length $10^2 + 1=101$characters that form the desired supersting.  


\begin{figure}[hb]
\centering
\begin{tikzpicture}[scale=0.75, transform shape]
  %the multiplication with floats is not possible. Thus I split the loop in two.
  \foreach \number in {0,...,4}{
      % Computer angle:
        \mycount=\number
  \multiply\mycount by 36
        \advance\mycount by 18
      \node[draw,circle,inner sep=0.25cm] (N-\number) at (\the\mycount:5.4cm) {$\number$};
    }
  \foreach \number in {5,...,9}{
      % Computer angle:
        \mycount=\number
  \multiply\mycount by 36
        \advance\mycount by 18
      \node[draw,circle,inner sep=0.25cm] (N-\number) at (\the\mycount:5.4cm) {$\number$};
    }
  \foreach \number in {0,...,8}{
        \mycount=\number
  \foreach \numbera in {\the\mycount,...,9}{
    \path (N-\number) edge[->,bend right=3] (N-\numbera)  edge[<-,bend
      left=3] (N-\numbera);
  }
  \foreach \number/\in\out in{0/0/45, 1/36/81, 2/72/115, 3/108/155, 4/144/191,5/180/216,6/216/252, 7/252/288,8/288/324,9/324/360}
  \draw (N-\number) .. controls +(\in:2cm) and +(\out:2cm) .. (N-\number);
}
\end{tikzpicture}
\caption{A Eulerian path approach to finding the shortest superstring problem of the set of 2 digit decimal numbers.  \tiny $\LaTeX$ code adapted from http://www.texample.net/tikz/examples/complete-graph/, by Jean-Noël Quintin}
\label{numgraph}
\end{figure}

\clearpage

\section{Problem 8.9}
A Eulerian graph corresponding to $S=\{\text{ATG, GGG, GGT, GTA, GTG, TAT, TGG}\}$ is 

\begin{figure}[h]
\centering
\begin{tikzpicture}[shorten >=1pt,->]
  \tikzstyle{vertex}=[circle,fill=black!25,minimum size=13pt,inner sep=0pt]

  \foreach \name/\x/\y in {AT/0/1,TA/0/-1,TG/3/1,GT/3/-1,GG/5/0}
    \node[vertex] (G-\name) at (\x,\y) {$\name$};

  \foreach \from/\to in {TA/AT,AT/TG,TG/GG,GG/GT,GT/TG,GT/TA}
    \draw[thick] (G-\from) -- (G-\to);

   \foreach \node/\in/\out in{GG/-18/18}
  \draw (G-\node) .. controls +(\in:2cm) and +(\out:2cm) .. (G-\node);

\end{tikzpicture}
\caption{The Eulerian Path approach to finding the shortest common superstring in problem 8.9.}
\label{eulerpath2}
\end{figure}

The three Eulerian paths of this graph correspond to GTATGGGTG and GTGGGTATG.  Both of these paths start at the GT node but travel in slightly different paths to visit all edges of the graph.  Interestingly, the strings are reversals of each other.

\clearpage
\section{PEPTID}

\begin{figure}[h]
\begin{center}
\includegraphics[width=5in]{aaweights.jpg}
\end{center}
\caption{The list of amino acid weights and codes from Lecture 15, slide 4.}
\label{aaweights}
\end{figure}

\paragraph{4a}  The theoretical MS/MS spectrum of PEPTID is shown below in Table~\ref{peptid}.\\
\begin{table}[hb]
\begin{tabular}{ccccccccc|c}
Wt & & & & & & & & Wt&Calculation\\\hline
760 &P&E&P&T&I&D&|&-&$115+147+115+119+131+133\hspace{.5cm}$\\
627 &P&E&P&T&I&|&D&133&$115+147+115+119+131\hspace{1cm}133$\\
496 &P&E&P&T&|&I&D&264&$115+147+115+119\hspace{1cm}131+133$\\
377 &P&E&P&|&T&I&D&383&$115+147+115\hspace{1cm}119+131+133$\\
262 &P&E&|&P&T&I&D&498&$115+147\hspace{1cm}115+119+131+133$\\
115 &P&|&E&P&T&I&D&645&$115\hspace{1cm}147+115+119+131+133$\\
\end{tabular}
\begin{center}
$S1=\{115,133,262,264,377,383,496,498,627,645,760\}$
\end{center}
\caption{The theoretical MS/MS spectrum of PEPTID, with calculations}
\label{peptid}
\end{table}

\paragraph{4b}  The Shared Peak Counts (SPC) of $S2$ and $S3$ with $S1$ are given below in Table~\ref{SPCtable}.
\begin{table}[hbt]
\begin{displaymath}
\begin{array}{ll}
S1=\{115,133,262,264,377,383,496,498,627,645,760\}& \\
S2=\{\textbf{115,133,264},280,\textbf{383}, 395 \textbf{498},514,\textbf{645},663,778\}& \text{SPC}=6\\
S3=\{\textbf{115,133},280, 337,395,456,514,571,718,736,851\}& \text{SPC}=2\\
\end{array}
\end{displaymath}
\caption{The Shared Peak Counts (SPC) of $S2$ and $S3$ with $S1$.  Values of $S2$ and $S3$ found in $S1$ are highlighted.}
\label{SPCtable}
\end{table}

\paragraph{4c}  The spectral convolutions of $S2$ and $S3$, respectively, with $S1$, are given in Figure~\ref{convol}.  
As can immediately be seen from the colorings, $S1$ and $S2$ are more similar by this method of comparison than are $S1$ and $S3$.  The $S1\ominus S2$ convolution has peak heights of 5 and 6, whereas the highest peak in $S1\ominus S3$ is only 4.
%Need to dig into calculations on p 292-293

\begin{figure}
\begin{center}
\begin{tabular}{cc}
\includegraphics[width=2.3in]{s2convol} & \includegraphics[width=2.3in]{s3convol}
\end{tabular}
\end{center}
\caption{The spectral convolutions of $S2$ and $S3$, respectively, with $S1$. D(k) for k>2 are highlighted.}
\label{convol}
\end{figure}

\paragraph{4d}  To determine the residue substitutions that might have given rise to $S2$ and $S3$, we'll need a difference matrix of all the amino acids.  This is given in Figure~\ref{aminodiff}.\\
\begin{figure}
\centering
\includegraphics[width=4in]{aminodiff.jpg}
\caption{A difference matrix of all the amino acids}
\label{aminodiff}
\end{figure}
The possible residues changes for $S1 \ominus S3$ with a mass difference of 18 are:
\begin{itemize}
  \item E $\rightarrow$ F  \hspace{1cm} PFPTID
  \item P $\rightarrow$ D  \hspace{1cm}DEDTID
  \item I $\rightarrow$ M   \hspace{1cm}PEPTMD
\end{itemize}

The only possible residues change for $S1 \ominus S3$ with a mass difference of 73 are:
\begin{itemize}
  \item I $\rightarrow$ W  \hspace{1cm} PEPTWD
\end{itemize}

We can combine these observations with the theoretical spectrum calculated above for $S1$, PEPTID, to determine the residue substitutions for $S3$.  Table~\ref{S3calcs} shows that PFPTWD is a possible residue substition for PEPTID consistent with spectrum $S3$.
\begin{table}[h]
\begin{tabular}{ccccccccc|ccccccccc}
Wt & & & & & & & & Wt&Wt & & & & & & & & Wt\\\hline
760 &P&E&P&T&I&D&|&-&851 &P&F&P&T&W&D&|&-\\
627 &P&E&P&T&I&|&D&133&718 &P&F&P&T&I&|&D&133\\
496 &P&E&P&T&|&I&D&264&514 &P&F&P&T&|&W&D&337\\
377 &P&E&P&|&T&I&D&383&395 &P&F&P&|&T&W&D&456\\
262 &P&E&|&P&T&I&D&498&280 &P&F&|&P&T&W&D&571\\
115 &P&|&E&P&T&I&D&645&115 &P&|&F&P&T&W&D&736\\
\end{tabular}
\caption{Calculations of a possible residue substation from PEPTID to PFPTWD}
\label{S3calcs}
\end{table}

The substitution picture with $S2$ is somewhat more complicated, if only because of the higher numbers of D($k$) for $k > 2$.  In the end, though, a single substitution, E $\rightarrow$ F, accounts for all of the observed spectral differences.  As can be seen in Figure~\ref{aminodiff},  F is the only acid with a weight difference of 18 with E.  
\begin{table}[h]
\begin{tabular}{ccccccccc|ccccccccc}
Wt & & & & & & & & Wt&Wt & & & & & & & & Wt\\\hline
760 &P&E&P&T&I&D&|&-&778 &P&F&P&T&I&D&|&-\\
627 &P&E&P&T&I&|&D&133&645 &P&F&P&T&I&|&D&133\\
496 &P&E&P&T&|&I&D&264&514 &P&F&P&T&|&I&D&264\\
377 &P&E&P&|&T&I&D&383&395 &P&F&P&|&T&I&D&383\\
262 &P&E&|&P&T&I&D&498&280 &P&F&|&P&T&I&D&498\\
115 &P&|&E&P&T&I&D&645&115 &P&|&F&P&T&I&D&663\\
\end{tabular}
\caption{Calculations of a possible residue substition from PEPTID to PFPTID}
\label{S2calcs}
\end{table}

\clearpage
\section{Programming Exercise}
See also electronic submission.
\begin{lstlisting}

from sys import exit
import networkx as nx
from random import choice

"""A program to find a Eulerian path through a directed graph"""

# Construct graph from input
# input=eval(raw_input("enter your graph"))
# G=nx.DiGraph()
# for edge in input:
# 	G.add_edge(edge[0],edge[1], label=edge[2])

n = 0

def in_out_balance(node):
	balance=G.in_degree(node) - G.out_degree(node)
	return balance
	
def set_semibalanced(node):
	global unbalanced
	global n
	G.node[node]['semibalanced'] = 1
	unbalanced = 1
	n = n + 1

def none():
	print "None"
	return None
	exit()

def find_balance(G):
	global start
	global finish
	global unbalanced
	unbalanced = 0
	n = 0
	for node in G:
		balance = in_out_balance(node)
		if abs(balance) > 1:
			none()
		if abs(balance) == 1:
			set_semibalanced(node)
			if balance == 1:
				finish=node	
			else:
				start=node
		if n > 2:
			none()
	# Connect start and finish if unbalanced
	# Otherwise, pick a random starting point	
	if unbalanced==1:
		G.add_edge(finish,start, label=None)
	else:
		start=choice(G.nodes())

def euler(G):
	global start
	global finish
	find_balance(G)
	path=[]
	if nx.is_eulerian(G):
		circuit=list(nx.eulerian_circuit(G, start))
		for edge in circuit:
			current=G.edge[edge[0]][edge[1]]['label']
			if current != None:
				path.append(current)
		if unbalanced==0:
			finish=circuit.pop()[1]
		answer=start, path, finish
		print answer
		
	else: 
		none()

euler(G)


\end{lstlisting}
\end{document}