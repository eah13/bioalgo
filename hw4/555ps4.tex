
%!TEX TS-program = xelatex
%!TEX encoding = UTF-8 Unicode

\documentclass[12pt]{article}
\usepackage{geometry}                % See geometry.pdf to learn the layout options. There are lots.
\geometry{letterpaper}                   % ... or a4paper or a5paper or ... 
\usepackage{amsmath}
\usepackage{graphicx}
\usepackage{amssymb}
\usepackage{listings}
\lstset{language=Octave,
  basicstyle=\footnotesize,tabsize=3,
commentstyle=\color{blue},
morecomment=[l]{\#}}
\usepackage{fontspec,xltxtra,xunicode}
\defaultfontfeatures{Mapping=tex-text}
\setromanfont[Mapping=tex-text]{Hoefler Text}
\setsansfont[Scale=MatchLowercase,Mapping=tex-text]{Gill Sans}
\setmonofont[Scale=MatchLowercase]{Andale Mono}

\author{Elliott Hauser}
\title{Comp 555 Problem Set 4}
\begin{document}
\maketitle
\section*{Question 1}
\subsection*{a}  While both Hash-table and Suffix trees would correctly solve this problem, they may be prefered over one another depending on the circumstances.  Suffix trees have a run time of $O(m)$, making them preferable \textit{ceteris paribus}.  But there is computational and engineering overhead to generate the suffix tree. Hash tables are implemented natively in many programming languages (such as Python's \lstinline{dict} structure), making them easier to implement in some situations.

That said, for our purposes computational and engineering overhead are negligible when compared to runtime savings, especially given that $n$ is likely to be very long.  So, the suffix tree structure is most appropriate for our purposes here.

\subsection*{b} Assuming the suffix table for $m$ is already constructed as a graph, an algorithm to find the number of occurrences of $m$ in $n$ is
\begin{lstlisting}[language=python, mathescape=true]

Preprocess $n$ into suffix tree $s_n$ in $O(n)$.

SuffixSearch($m$, $s_n$)
position = 0
for edge in childEdge($s_n$) == $m$[position:len(edge_label)]
	if descendent vertex has child edge == $m$[position +1:len(child_edge_label)]
		SuffixSearch($m$[position+1:len($m$)], descendants(edge)
	else
		matches = count(descendants(edge))
		return matches

\end{lstlisting}

\subsection*{c}  The time complexity is %insert precise calculation 
$O(m)$.

\section*{Question 2}
% Hash Table Scheme
\subsection*{a} %Hash function

\subsection*{b} %Hash table

\subsection*{c} %Students with score = 4
\section*{Question 3}

\section*{Question 4}

\section*{Question 5}

\section*{Question 6: Programming exercise}


\end{document} 